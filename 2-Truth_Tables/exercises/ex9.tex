Use truth tables to determine which of these statements are tautologies, which are contradictions, and which are neither:
\begin{enumerate}[label=(\alph*)]
    \item $(P \vee Q) \wedge (\neg P \vee \neg Q).$
    \item $(P \vee Q) \wedge (\neg P \wedge \neg Q).$
    \item $(P \vee Q) \vee (\neg P \vee \neg Q).$
    \item $[P \wedge (Q \vee \neg R)] \vee (\neg P \vee R).$
\end{enumerate}

\textbf{Solution:} Recall that a contradiction is always false for all combination of truth values for its inputs. Similarly, a tautology is always true for all combination of truth values for its inputs.
Consider the following substitutions.
\begin{itemize}
    \item $A$: $(P \vee Q) \wedge (\neg P \vee \neg Q)$.
    \item $B$: $(P \vee Q) \wedge (\neg P \wedge \neg Q)$.
    \item $C$: $(P \vee Q) \vee (\neg P \vee \neg Q)$.
    \item $D$: $[P \wedge (Q \vee \neg R)] \vee (\neg P \vee R)$.
\end{itemize}

\[
\begin{array}{|cc||c|c|c|}
\hline
P & Q & A & \color{red}{B} & \color{blue}{C} \\
\hline
T & T & F & \color{red}{F} & \color{blue}{T} \\
T & F & T & \color{red}{F} & \color{blue}{T} \\
F & T & T & \color{red}{F} & \color{blue}{T} \\
F & F & F & \color{red}{F} & \color{blue}{T} \\
\hline
\end{array}
\]

We have already determined that $A$ is neither a tautology or contradiction; $B$ is a contradiction, $C$ is a tautology. \

Finally, we take on $D$ exclusively and the truth table shows it is a tautology.
\[
\begin{array}{|ccc||ccc||c|}
\hline
P & Q & R & Q \vee \neg R & P \wedge (Q \vee \neg R) & \neg P \vee R & [P \wedge (Q \vee \neg R)] \vee (\neg P \vee R)\\
F & F & F & T & F & T & T \\
F & F & T & F & F & T & T \\
F & T & F & T & F & T & T \\
F & T & T & T & F & T & T \\
T & F & F & T & T & F & T \\
T & F & T & F & F & T & T \\
T & T & F & T & T & F & T \\
T & T & T & T & T & T & T \\
\hline
\end{array}
\]

\pagebreak