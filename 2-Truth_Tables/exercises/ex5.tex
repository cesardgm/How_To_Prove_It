Some mathematicians use the symbol $\downarrow$ to mean \textit{nor}. In other words, $P \downarrow Q$ means "neither $P$ nor $Q$".
\begin{enumerate}[label=(\alph*)]
    \item Make a truth table for $P \downarrow Q$.
    \item Find a formula using only the connectives $\wedge$, $\vee$, and $\neg$ that is equivalent to $P \downarrow Q$.
    \item Find formulas using only the connective $\downarrow$ that are equivalent to $\neg P$, $P \vee Q$, and $P \wedge Q$.
\end{enumerate}

\textbf{Solution:}
\begin{enumerate}[label=(\alph*)]
\item 
    \[
    \begin{array}{|c|c|c|}
    \hline
    P & Q & P \downarrow Q \\
    \hline
    \text{False} & \text{False} & \text{True} \\
    \text{False} & \text{True} & \text{False} \\
    \text{True} & \text{False} & \text{False} \\
    \text{True} & \text{True} & \text{False} \\
    \hline
    \end{array}
    \]

\item The formula $\neg P \wedge \neg Q$ is equivalent to $P \downarrow Q$ since for every possible combination of truth values for the propositions, the truth values of both formulas are the same. 
\[
\begin{array}{|c|c|c|c|c||c|}
\hline
P & Q & \neg P & \neg Q & \neg P \wedge \neg Q & P \downarrow Q \\
\hline
\text{False} & \text{False} & \text{True} & \text{True} & \text{True} & \text{True} \\
\text{False} & \text{True} & \text{True} & \text{False} & \text{False} & \text{False} \\
\text{True} & \text{False} & \text{False} & \text{True} & \text{False} & \text{False} \\
\text{True} & \text{True} & \text{False} & \text{False} & \text{False} & \text{False} \\
\hline
\end{array}
\]

\pagebreak

\item \begin{itemize}
    \item The formula $P \downarrow P$ is equivalent to $\neg P$.
    \[
    \begin{array}{|c|c|c||c|}
    \hline
    P & P & P \downarrow P & \neg P \\
    \hline
    \text{False} & \text{False} & \text{True} & \text{True}\\
    \text{True} & \text{True} & \text{False} & \text{False} \\
    \hline
    \end{array}
    \]

    \item The formula $(P \downarrow Q) \downarrow (P \downarrow Q)$ is equivalent to $\neg (P \downarrow Q)$ (using the result from above, i.e., $\neg P \equiv P \downarrow P$), hence equivalent to $P \vee Q$. We justify the answer with our truth table.
    \[
    \begin{array}{|c|c|c|c||c|}
    \hline
    P & Q & P \downarrow Q & \neg (P \downarrow Q) & P \vee Q \\
    \hline
    \text{False} & \text{False} & \text{True} & \text{False} & \text{False}\\
    \text{False} & \text{True} & \text{False} & \text{True} & \text{True} \\
    \text{True} & \text{False} & \text{False} & \text{True} & \text{True} \\
    \text{True} & \text{True} & \text{False} & \text{True} & \text{True} \\
    \hline
    \end{array}
    \]

    \item The formula $(P \downarrow P) \downarrow (Q \downarrow Q)$ is equivalent to $P \wedge Q$.
    \begin{alignat*}{2}
        P \wedge Q & \equiv \neg \neg P \wedge \neg \neg Q && \quad \text{(Double Negation Law)} \\
                   & \equiv \neg (\neg P \vee \neg Q) && \quad \text{(DeMorgan's Law)} \\
                   & \equiv \neg \neg (\neg P \downarrow \neg Q) && \quad \text{(Results Above: $\neg(P \downarrow Q) \equiv P \vee Q$)} \\
                   & \equiv \neg P \downarrow \neg Q && \quad \text{(Double Negation Law)} \\
                   & \equiv (P \downarrow P) \downarrow (Q \downarrow Q) && \quad \text{(Results Above: $\neg P \equiv P \downarrow P$)}
    \end{alignat*}
    We also justify our answer with a truth table.
    \[
    \begin{array}{|c|c|c|c|c||c|}
    \hline
    P & Q & P \downarrow P & Q \downarrow Q & (P \downarrow P) \downarrow (Q \downarrow Q) & P \wedge Q \\
    \hline
    \text{True} & \text{True} & \text{False} & \text{False} & \text{True} & \text{True} \\
    \text{True} & \text{False} & \text{False} & \text{True} & \text{False} & \text{False} \\
    \text{False} & \text{True} & \text{True} & \text{False} & \text{False} & \text{False} \\      
    \text{False} & \text{False} & \text{True} & \text{True} & \text{False} & \text{False} \\
    \hline
    \end{array}
    \]
\end{itemize}
\end{enumerate}

\pagebreak