% This is exercise.tex
% Ex 3
The proof of Theorem 3 gives a method for finding a prime number different from any in a given list of prime numbers.

\begin{enumerate}[label=(\alph*)]
    \item Use this method to find a prime different from 2,3,5, and 7.
    \item Use this method to find a prime different from 2,5, and 11.
\end{enumerate}

\textbf{Solution:}
Recall that \textbf{Theorem 3} gave the following method for finding a prime number different from those in the given list of prime numbers in its premise. If $m=p_1p_2\cdot \cdot \cdot p_n+1$, where $p_1,p_2,..., p_n$ is a list of prime numbers and $m$ is a positive integer, then $m$ is a prime or a product of primes.
\begin{enumerate}[label=(\alph*)]
   \item If $p_1p_2\cdot \cdot \cdot p_n = (2)(3)(5)(7) $, then $m=(2)(3)(5)(7)+1=211$. Note that 211 is prime and not in the list.
   \item If $p_1p_2\cdot \cdot \cdot p_n = (2)(5)(11) $, then $m=(2)(5)(11)+1=111=(3)(37)$. Note that 3 and 37 are prime and not in the list.
\end{enumerate}

\pagebreak