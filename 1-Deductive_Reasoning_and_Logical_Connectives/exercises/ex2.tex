% This is exercise.tex
% Ex 2
Analyze the logical forms of the following statements:
\begin{enumerate}[label=(\alph*)]
    \item Either John and Bill are both telling the truth, or neither of them is.
    \item I'll have either fish or chicken, but I won't have both fish and mashed potatoes.
    \item 3 is a common divisor of 6,9, and 15.
\end{enumerate}

\textbf{Solution:}
\begin{enumerate}[label=(\alph*)]
    \item Let $J$ represent the statement "John is telling the truth", and $B$ represent the statement "Bill is telling the truth". The given statement can be thus translated into propositional logic as follows:
    $$(J \wedge B) \vee (\neg J \wedge \neg B).$$

    \item Let $F$ represent the statement "I will have fish", $C$ represent the statement "I will have chicken" and $P$ represent the statement "I will have mashed potatoes". The given statement can be thus translated into propositional logic as follows:
    $$(F \vee C) \wedge \neg (F \wedge P).$$

    \item The statement "3 is a common divisor of 6,9, and 15" can be understood as "3 is a common divisor of 6", and "3 is a common divisor of 9", and "3 is a common divisor of 15". It can be formally written as:
    $$(3 | 6) \wedge (3 | 9) \wedge (3 | 15).$$
\end{enumerate}

\pagebreak