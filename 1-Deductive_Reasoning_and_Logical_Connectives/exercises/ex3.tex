% This is exercise.tex
% Ex 3
Analyze the logical forms of the following statements:
\begin{enumerate}[label=(\alph*)]
    \item Alice and Bob are not both in the room.
    \item Alice and Bob are both not in the room.
    \item Either Alice or Bob is not in the room.
    \item Neither Alice nor Bob is in the room.
\end{enumerate}

\textbf{Solution:}
Let $A$ represent the statement "Alice is in the room" and $B$ the statement "Bob is in the room". The given statements can be thus translated into propositional logic as follows:

\begin{enumerate}[label=(\alph*)]
    \item It is not the case that both Alice is in the room and Bob is in the room at the same time: $\neg (A \wedge B)$
    \item Both Alice is not in the room, and Bob is not in the room. There are no other possibilities; both individuals are absent from the room: $\neg A \wedge \neg B$
    \item Either Alice is not in the room, or Bob is not in the room, or both are not in the room: $\neg A \vee \neg B$ 
    \item Alice is not in the room, and Bob is also not in the room. There is no scenario where either Alice or Bob is present in the room; both are absent: $\neg A \wedge \neg B$
\end{enumerate}

\begin{enumerate}
    \item[NOTE] \textbf{"Neither Alice nor Bob is in the room"} uses a construction that directly states that both individuals are absent from the room. It's a single statement that negates the presence of both individuals simultaneously.
    \item[] \textbf{"Alice and Bob are both not in the room"} effectively communicates the same information, indicating that both Alice and Bob are absent from the room. This construction explicitly states the absence of each individual.
    \item[] Logically, both sentences assert the absence of both Alice and Bob from the room. In terms of logical operators, both sentences can be understood as expressing a conjunction (logical AND) of two negations.
\end{enumerate}

\pagebreak