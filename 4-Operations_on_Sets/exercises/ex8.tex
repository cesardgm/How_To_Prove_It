For each of the following sets, write out (using logical symbols) what it means for an object $x$ to be an element of the set. Then determine which of these sets must be equal to each other by determining which statements are equivalent.

\begin{enumerate}[label=(\alph*)]
    \item $(A \mybackslash B) \mybackslash C$.
    \item $A \mybackslash (B \mybackslash C)$.
    \item $(A \mybackslash B) \cup (A \cap C)$.
    \item $(A \mybackslash B) \cap (A \mybackslash C)$.
    \item $A \mybackslash (B \cup C)$.
\end{enumerate}

\textbf{Solution:}
\begin{enumerate}[label=(\alph*)]
    \item Below we prove that $(A \mybackslash B) \mybackslash C \equiv A \mybackslash (B \cup C)$, hence $\textbf{(a)}$ and $\textbf{(e)}$ are equivalent. We will see later that \textbf{(d)} is also equivalent.
    \begin{alignat*}{2}
        x \in (A \mybackslash B) \mybackslash C &\equiv (x \in A \mybackslash B) \wedge \neg (x \in C) && \quad \textbf{(Def. of Difference of Sets)} \\
        &\equiv [x \in A \wedge \neg (x \in B)] \wedge \neg (x \in C) && \quad \textbf{(Def. of Difference of Sets)} \\
        &\equiv x \in A \wedge [\neg (x \in B) \wedge \neg (x \in C)] && \quad \textbf{(Associative Law)} \\
        &\equiv x \in A \wedge \neg (x \in B \vee x \in C) && \quad \textbf{(DeMorgan's Law)} \\
        &\equiv x \in A \wedge \neg (x \in B \cup C) && \quad \textbf{(Def. of Union of Sets)} \\
        &\equiv x \in A \mybackslash (B \cup C) && \quad \textbf{(Def. of Difference of Sets)} \\
    \end{alignat*}
    
    \item Below we prove that $A \mybackslash (B \mybackslash C) \equiv A \mybackslash B) \cup (A \cap C)$, hence $\textbf{(b)}$ and $\textbf{(c)}$ are equivalent.
    \begin{alignat*}{2}
        x \in A \mybackslash (B \mybackslash C) &\equiv x \in A \wedge \neg (x \in B \mybackslash C) && \quad \textbf{(Def. Difference of Sets)}\\
        &\equiv x \in A \wedge \neg [x \in B \wedge \neg (x \in C)] && \quad \textbf{(Def. Difference of Sets)}\\
        &\equiv x \in A \wedge [\neg (x \in B) \vee x \in C] && \quad \textbf{(DeMorgan's Law)}\\
        &\equiv [x \in A \wedge \neg (x \in B)] \vee (x \in A \wedge x \in C) && \quad \textbf{(Distributive Law}\\
        &\equiv (x \in A \mybackslash B) \vee (x \in A \wedge x \in C) && \quad \textbf{(Def. Difference of Sets)}\\
        &\equiv (x \in A \mybackslash B) \vee (x \in A \cap C) && \quad \textbf{(Def. Intersection of Sets)}\\
        &\equiv x \in (A \mybackslash B) \cup (A \cap C) && \quad \textbf{(Def. Union of Sets)}\\
    \end{alignat*}
    
    \item See \textbf{(b)}.
    \pagebreak
    
    \item Below we prove that $(A \mybackslash B) \cap (A \mybackslash C) \equiv A \mybackslash (B \cup C)$. But we showed in \textbf{(a)} that $A \mybackslash (B \cup C) \equiv (A \mybackslash B) \mybackslash C$. Hence, \textbf{(a)}, \textbf{(d)} and \textbf{(e)} are equivalent.
    \begin{alignat*}{2}
        x \in (A \mybackslash B) \cap (A \mybackslash C) &\equiv (x \in A \mybackslash B) \wedge (x \in A \mybackslash C) && \quad \textbf{(Def. Intersection of Sets)}\\
        &\equiv [x \in A \wedge \neg (x \in B)] \wedge (x \in A \mybackslash C) && \quad \textbf{(Def. Difference of Sets)}\\
        &\equiv [x \in A \wedge \neg (x \in B)] \wedge [x \in A \wedge \neg (x \in C)] && \quad \textbf{(Def. Difference of Sets)}\\
        &\equiv x \in A \wedge \neg (x \in B) \wedge x \in A \wedge \neg (x \in C) && \quad \textbf{(Associative Law)}\\
        &\equiv x \in A \wedge x \in A \wedge \neg (x \in B) \wedge \neg (x \in C) && \quad \textbf{(Commutitative Law)}\\
        &\equiv (x \in A \wedge x \in A) \wedge \neg (x \in B) \wedge \neg (x \in C) && \quad \textbf{(Associative Law)}\\
        &\equiv x \in A \wedge \neg (x \in B) \wedge \neg (x \in C) && \quad \textbf{(Idempotent Law)}\\
        &\equiv x \in A \wedge \neg (x \in B \vee x \in C) && \quad \textbf{(DeMorgan's Law)}\\
        &\equiv x \in A \wedge \neg (x \in B \cup C) && \quad \textbf{(Def. Union of Sets)}\\
        &\equiv x \in A \mybackslash (B \cup C) && \quad \textbf{(Def. Difference of Sets)}\\
    \end{alignat*}
    
    \item See \textbf{(a)}.
\end{enumerate}
\pagebreak