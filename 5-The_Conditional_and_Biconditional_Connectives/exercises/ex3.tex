Analyze the logical form of the following statement:
\begin{enumerate}[label=(\alph*)]
    \item If it is raining, then it is windy and the sun is not shining.

    Now analyze the following statements. Also, for each statement determine whether the statement is equivalent to either statement (a) or its converse.

    \item It is windy and not sunny only if it is raining.
    \item Rain is a sufficient condition for wind with no sunshine.
    \item Rain is a necessary condition for wind with no sunshine.
    \item It's not raining, if either the sun is shining or it's not windy.
    \item Wind is a necessary condition for it to be rainy, and so is a lack of sunshine.
    \item Either it is windy only if it is raining, or it is not sunny only if it is raining.
\end{enumerate}

\textbf{Solution:}
\begin{enumerate}[label=(\alph*)]
    \item Let the following symbols represent the corresponding statements:
        \begin{itemize}
        \item $R$: It is raining
        \item $W$: It is windy
        \item $S$: The sun is shining
        \end{itemize}
        The logical form of the original statement can be expressed as:
        $$R \rightarrow (W \wedge \neg S)$$
        The converse of the statement can be expressed as:
        $$(W \wedge \neg S) \rightarrow R $$
        If we let $P = (W \wedge \neg S)$ and $Q = R$, then the converse can be stated in various equivalent ways:
        \begin{itemize}
        \item If it is windy and the sun is not shining, then it is raining. (P implies Q)
        \item It is raining, if it is windy and the sun is not shining. (Q, if P)
        \item It is windy and the sun is not shining only if it is raining. (P only if Q)
        \item Being windy and the sun not shining is a sufficient condition for it to be raining. (P is a sufficient condition for Q)
        \item That it is raining is a necessary condition for it to be windy and the sun not to shine. (Q is a necessary condition for P)
        \item It is a necessary condition for it to be windy and the sun not shining that it is raining (Exercise 1d).
        \end{itemize}
    \pagebreak
    
    \item "It is windy and not sunny only if it is raining" can be expressed as: $$(W \wedge \neg S) \rightarrow R.$$
    \item "Rain is a sufficient condition for wind with no sunshine" can be expressed as: $$R \rightarrow (W \wedge \neg S).$$
    \item "Rain is a necessary condition for wind with no sunshine" can be expressed as: $$(W \wedge \neg S) \rightarrow R.$$
    \item "It's not raining, if either the sun is shining or it's not windy" can be expressed as: 
        \begin{alignat*}{2}
        (\neg W \vee S) & \rightarrow \neg R && \quad \textbf{Conditional statement}\\
        \neg (\neg W \vee S) & \vee \neg R && \quad \textbf{Conditional Law}\\
        (W \wedge \neg S) & \vee \neg R && \quad \textbf{DeMorgan's Law}\\
        \neg R &\vee (W \wedge \neg S) && \quad \textbf{Commutative Law}\\
        R &\rightarrow (W \wedge \neg S) && \quad \textbf{Conditional Law}\\
        \end{alignat*}
    \item "Wind is a necessary condition for it to be rainy, and so is a lack of sunshine" can be expressed as: 
        \begin{alignat*}{2}
            (R \rightarrow W) &\wedge (R \rightarrow \neg S) && \quad \textbf{Compound sentence} \\   
            (\neg R \vee W) &\wedge (\neg R \vee \neg S) && \quad \textbf{Conditional Law} \\  
            \neg R &\vee (W \wedge \neg S) && \quad \textbf{Distributive law} \\  
            R &\rightarrow (W \wedge \neg S) && \quad \textbf{Conditional Law} \\
        \end{alignat*}
    
    \item "Either it is windy only if it is raining, or it is not sunny only if it is raining" can be expressed as:
        \begin{alignat*}{2}
            (W \rightarrow R) &\vee (\neg S \rightarrow R) && \quad \textbf{Compound sentence}\\
            (\neg W \vee R) &\vee (S \vee R) && \quad \textbf{Conditional Law}\\
            \neg W \vee R &\vee S \vee R && \quad \textbf{Associative law}\\
            R \vee R &\vee \neg W \vee S && \quad \textbf{Commutative law}\\
            (R \vee R) &\vee (\neg W \vee S) && \quad \textbf{Associative law}\\
            R &\vee (\neg W \vee S) && \quad \textbf{Idempotent law}\\
            \neg R &\rightarrow (\neg W \vee S) && \quad \textbf{Conditional Law}\\
            (W \wedge \neg S) &\rightarrow R && \quad \textbf{Contrapositive}\\
        \end{alignat*}
\end{enumerate}

We observe that the original statement \textbf{3a} is logically equivalent with statements \textbf{3c, 3e and 3f}. \\
We observe that the converse of statement \textbf{3a} is logically equivalent with statements \textbf{3b, 3d, and 3g}.

\pagebreak