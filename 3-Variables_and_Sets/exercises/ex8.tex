What are the truth sets of the following statements? List a few elements of the truth set if you can.

\begin{enumerate}[label=(\alph*)]
    \item $x$ is a real number and $x^2-4x+3=0$.
    \item $x$ is a real number and $x^2-2x+3=0$.
    \item $x$ is a real number and $5 \in \{y \in \mathbb{R} \ | \ x^2+y^2 < 50 \}$.
\end{enumerate}

\textbf{Solution:}
\begin{enumerate}[label=(\alph*)]
    \item To find the truth set, we need to solve the equation $x^2-4x+3=0$. Using the quadratic formula, we get:
    \begin{align*}
        x &= \frac{-b \pm \sqrt{b^2-4ac}}{2a} \\
          &= \frac{4 \pm \sqrt{(-4)^2-4(1)(3)}}{2(1)} \\
          &= \frac{4 \pm \sqrt{4}}{2} \\
          &= \frac{4 \pm 2}{2} \\
          &= 1 \text{ or } 3
    \end{align*}
    Therefore, the truth set is:
    \begin{align*}
        \{ x \ | \ &x \text{ is a real number and } x^2-4x+3=0 \} \\
        &\equiv \{1, 3\}
    \end{align*}
    
    \item Similarly, solving the equation $x^2-2x+3=0$ using the quadratic formula:
    \begin{align*}
        x &= \frac{-b \pm \sqrt{b^2-4ac}}{2a} \\
          &= \frac{2 \pm \sqrt{(-2)^2-4(1)(3)}}{2(1)} \\
          &= \frac{2 \pm \sqrt{-8}}{2}
    \end{align*}
    Since $\sqrt{-8}$ is not a real number, there are no real solutions to this equation. Therefore, the truth set is:
    \begin{align*}
        \{ x \ | \ &x \text{ is a real number and } x^2-2x+3=0 \} \\
        &\equiv \{\}
    \end{align*}
    
    \item The statement "$5 \in \{y \in \mathbb{R} \ | \ x^2+y^2 < 50 \}$" is equivalent to "$x^2+5^2 < 50$". Solving this inequality:
    \begin{align*}
        x^2+5^2 &< 50 \\
        x^2+25 &< 50 \\
        x^2 &< 25 \\
        -5 < x &< 5
    \end{align*}
    Therefore, the truth set is:
    \begin{align*}
        \{ x \ | \ &x \text{ is a real number and } 5 \in \{y \in \mathbb{R} \ | \ x^2+y^2 < 50 \} \} \\
        &\equiv \{ x \ | \ x \in \mathbb{R} \text{ and } -5 < x < 5 \}
    \end{align*}
    Some elements of this truth set include -4.9, 0, 3.14, etc.
\end{enumerate}